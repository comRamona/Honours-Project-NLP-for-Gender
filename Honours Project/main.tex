
% to choose your degree
% please un-comment just one of the following
\documentclass[bsc,frontabs,twoside,singlespacing,parskip,deptreport]{infthesis}     % for BSc, BEng etc.
\usepackage[T1]{fontenc}
\usepackage{amssymb,amsmath}
\usepackage{txfonts}
\usepackage{microtype}

% For figures
\usepackage{graphicx}
\usepackage{subfigure} 

% For citations
\usepackage{natbib}

% For algorithms
\usepackage{algorithm}
\usepackage{algorithmic}

% the hyperref package is used to produce hyperlinks in the
% resulting PDF.  If this breaks your system, please commend out the
% following usepackage line and replace \usepackage{mlp2017} with
% \usepackage[nohyperref]{mlp2017} below.
\usepackage{hyperref}
\usepackage{url}
\urlstyle{same}

% Packages hyperref and algorithmic misbehave sometimes.  We can fix
% this with the following command.
\newcommand{\theHalgorithm}{\arabic{algorithm}}
\begin{document}

\title{Why is AI "a sea of dudes"? Using data science and NLP methods to understand gender imbalance in a scientific community.}

\author{Ramona Comanescu}

% to choose your course
% please un-comment just one of the following
\course{Computer Science}

% please un-comment just one of the following
\project{Undergraduate Dissertation} % CS&E, E&SE, AI&L

\project{4th Year Project Report}

\date{\today}

\abstract{
This dissertation carries an in-depth study of gender in the field of computation linguistics. There has been an undeniable raise of deep learning methods in every field, including computational linguistics. We investigate how machine learning affects the gender gap in natural language processing. We perform name classification on a corpus of computational linguistic papers. There are three main topic we analyse: publication patterns, topic modeling and stylometry. We start by comparing career progression of men an women: we find that they have similar collaboration networks, but women's publishing patterns tend to be more sparse. We employ topic modeling to capture the shift of approaches in the field towards deep learning and contrast these with earlier findings.
}

\maketitle

\section*{Acknowledgements}
% I would like to thank my superviser, Dr. Adam Lopez for his unvaluable time and guidance during my project.
% My parents for the unexpected help with labeling names for gender and for the immense support. 
%Francisco Vargas for the insightful brainstorming, especially regarding name classification and helping me understand details of LDA.

\tableofcontents

%\pagenumbering{arabic}


\chapter{Introduction}
\section{Motivation}
Artificial intelligence (AI) has been growing rapidly in recent years. However, just 13\% percent of one of 2015's biggest AI conferences (NIPS) were women, according to Bloomberg. \cite{seaofdudes}
Microsoft researcher Margaret Mitchell calls Artificial Intelligence "a sea of dudes". \cite{seaofdudes} 
\section{Outline}
The main goal of this thesis is to analyse gender imbalance in the field of computational linguistics. We go beyond surface statistics and try to understand gender from multiple points of views.

\begin{itemize}
\item In chapter \ref{ch:publ} we compare publishing patterns of men and women. 
    \begin{itemize}
    \item Single authored papers and coauthorship
    \item Collaboration network
    \item 
    \end{itemize}
\end{itemize}

\chapter{Background}

\section{Previous work}

Most studies of gender balance in computer science are based on reporting statistics on enrolment. \cite{Jurafsky_Hesaid} go beyond the surface and study the difference in topics that different genders write about. They perform a "mostly-manual annotation" of the gender of each author, which we also use as a starting point for our name classification. They use Latent Dirichlet Allocation(LDA) topic models \citep{Blei_LDA} to study the difference in topics genders write about. Their study is conducted on the Association of Computational Linguistics Anthology Network (AAN) corpus \citep{aan}, using papers from 1965 to 2008. They find that "women publish more on dialogue, discourse, and sentiment, while men publish more than women in parsing, formal semantics and finite state models". They also find that the participation of females in the field has been steadily increasing.

 The resulting picture closely resembles and adds to that arrived at through qualitative analysis, showing that this form of topic modeling could be useful for sifting through datasets that had not previously been subject to any analysis. The strength of the relationships between topics illustrates the fluid and ubiquitous nature of sexism, with no single experience being unrelated to another.

Topic models for text corpora comprise a popular family of methods that have inspired many extensions to encode properties such as sparsity, interactions with covariates, and the gradual evolution of topics.

\section{Why gender balance matters}

\section{Ethical concerns}

\chapter{Methodology}

\section{Corpus}
\subsection{ACL Anthology Network Corpus}
We use The ACL anthology network corpus \cite{aan} with the lastest release being 2014. 

The  ACL 2012 Contributed Task \cite{contributed} outlines some of the problems that make maintaining an up to date and error-free ACL collection difficult. We make our own attempt at cleaning and preprocessing the data, before starting doing any analysis.
We remove all duplicate ids(papers published in multiple publication) and we are left with 22944 papers. 

The source dataset describes both automatic conversion from pdf to text and OCR techniques. However, we discover files for which conversion has failed, spitting out random character.
In order to detect malfunctioning conversion, we do automatic language detection, using \textbf{langid.py.} \cite{langid}
Pdf to text conversion only works for born digital pdf papers. Once we identify the files for which pdftotext failed, we used OCR with Tesseract. In the process, we also discover some papers where in french and we need to discard them. We eliminate 515 corrupted papers.
\subsection{ArXiv Corpus}

\section{Name classification}
\subsection{Methods}
\subsection{Evaluation}

\chapter{Publishing patterns}
\label{ch:publ}
\section{Summary statistics}
\section{Single authored papers and coauthorship}
\section{Collaboration network}

\chapter{Topic modelling}
\label{ch:topics}
\section{Latent Dirichlet Allocation}
% what it is
\section{Author-topic models}
\section{Experiments}
\section{Results}

\chapter{Stylometry}
\label{ch:style}
\subsection{Features}

\chapter{Conclusion}
As there is limited quantitative analysis of women and
men's scholarly output computer science(most analysis just report statistics on the number of graduates), this project
fills a gap in our knowledge about sex and publication productivity. We also analysed sex differences in single authorship and co-authorship
of journal articles.

\bibliographystyle {plainnat}
\bibliography{mybibfile}


\end{document}

